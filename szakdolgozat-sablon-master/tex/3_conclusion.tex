\chapter{Összefoglaló}
\label{chap:conclusion}

A szakdolgozatom elején bemutattam a Django keretrendszer által nyújtott lehetőségeket, az MVT architektúra működését, valamint a keretrendszer által biztosított eszközöket, amelyek megkönnyítik a webalkalmazások fejlesztését. Ismertettem a Django ORM működését, amely lehetővé teszi az adatbázis-kezelést SQL ismeretek nélkül, valamint bemutattam az alkalmazás adatbázisának FHIR szabvány szerinti felépítését.

Ezt követően részletesen bemutattam az alkalmazás felhasználói felületeit, amelyek a páciensek, orvosok és adminisztrátorok számára készültek. Ismertettem az időpontfoglalási rendszer működését, amely lehetővé teszi a páciensek számára, hogy egyszerűen foglaljanak időpontot a rendelőben elérhető kezelésekre. Bemutattam az online fizetési rendszer integrációját, amely a PayPal API segítségével biztosít biztonságos és egyszerű fizetési lehetőséget.

A szakdolgozatban külön fejezetet szenteltem az adminisztrátori felületek ismertetésére, amelyek lehetővé teszik a kezelések, orvosok és felhasználók adatainak kezelését. Részletesen bemutattam az orvosok munkaidő-beállítási lehetőségeit, valamint a páciensek kezeléstörténetének megtekintésére és szerkesztésére szolgáló funkciókat.

A dolgozatban kitértem az alkalmazás reszponzív designjára, amely modern CSS és JavaScript megoldásokkal készült, biztosítva a könnyű használatot különböző eszközökön. Bemutattam a világos és sötét mód közötti váltás megvalósítását, amely a felhasználói élményt hivatott javítani.

A szakdolgozat elkészítése során betekintést nyertem a Django keretrendszerben való fejlesztés rejtelmeibe, valamint a webfejlesztés különböző aspektusaiba. Megismertem az online fizetési rendszerek integrációjának folyamatát, az adatbázis-tervezés kihívásait, valamint a felhasználói jogosultságok kezelésének megvalósítását. A dolgozatom forráskódja elérhető a GitHubon, ahol részletes dokumentáció is található a projekt működéséről és telepítéséről.
