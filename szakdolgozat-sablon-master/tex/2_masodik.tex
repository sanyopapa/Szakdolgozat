\chapter{Alkalmazás struktúrája}
\label{chap:fejezet2}

Egy Django projekt esetében a projekt felépítése modulárisan, egy vagy több alkalmazásból (app) áll, melyek mindegyike egy adott funkcionális területért felel. A szakdolgozatom esetében a "rendelo" mappa tartalmazza a teljes webalkalmazás forráskódját.\\
A "rendelo" mappa a következő részekből áll:

\begin{itemize}
	\item Gyökérszint:
	\begin{itemize}
		\item manage.py: A Django projekt parancssori kezelője, amely a fejlesztési feladatok (például migrációk futtatása, szerver indítása) végrehajtását segíti.
		\item db.sqlite3: Az alapértelmezett, fejlesztési környezetben használt SQLite adatbázis fájlja.
	\end{itemize}
	\item Projekt főkönyvtára ("rendelo"): Itt találhatók a projekt globális beállításait és konfigurációs fájljait, mint például a settings.py, urls.py, wsgi.py és asgi.py. Ezek a fájlok felelősek az alkalmazás működésének alapvető paramétereinek meghatározásáért, az útvonalak kezeléséért és a szerverrel való kommunikációért.
	\item Alkalmazás könyvtára ("rendeloweboldal"): Ez a rész tartalmazza a rendszer egyes moduljait, amelyek a következő fő komponensekből állnak:
	\begin{itemize}
		\item models.py: Az adatbázis szerkezetét definiáló model-ek, melyek meghatározzák a páciensek, orvosok, kezelések, időpontfoglalások és az időpont foglalások fizetési státuszának struktúráját.
		\item views.py: A felhasználói kérések feldolgozásáért és az üzleti logika megvalósításáért felelős fájl, amely összeköti a modelleket a sablonokkal.
		\item forms.py: Az űrlapok és azok validációs szabályainak definíciója, melyek révén az adatbevitel és ellenőrzés történik.
		\item urls.py: Az alkalmazás specifikus URL-konfigurációja, amely a különböző view-k elérését biztosítja.
		\item admin.py: A Django beépített adminisztrátori felület konfigurációját tartalmazza. Hozzá kell adni az összes model-t az adatbázisból, amit elérhetővé szeretnénk tenni rajta.
		\item migrations/: Az adatbázis változásait követő migrációs fájlokat tartalmazza, dokumentálva a model-ek módosításait.
		\item static/ és templates/: A statikus fájlokat (CSS, JavaScript, és az alkalmazás designjához tartozó képek) illetve a HTML template-eket rendszerezi, amelyek a felhasználói felület megjelenítéséért felelősek.
	\end{itemize}
\end{itemize}

A projekt kialakítása moduláris és átlátható, mely lehetővé teszi a fejlesztés, karbantartás és bővítés egyszerű kezelését.
Emellett a projekt verziókezelése a GitHubon történik, így könnyen nyomon követhető az egész alkalmazás fejlesztése.