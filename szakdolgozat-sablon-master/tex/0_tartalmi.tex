\chapter*{Tartalmi összefoglaló}
\addcontentsline{toc}{section}{Tartalmi összefoglaló}

\noindent\textbf{A téma megnevezése:}

\noindent Fogászati rendelő webalkalmazás implementálása Django keretrendszerben

\noindent\textbf{A megadott feladat megfogalmazása:}

\noindent A Django keretrendszer által nyújtott előnyök megismerése, felhasználása a fogászati rendelő időpontfoglalási rendszerének fejlesztéséhez. Az alkalmazásnak rendelkeznie kell felhasználói, orvosi, és adminisztrátori felülettel. Meg kell valósítani az online időpontfoglalás, és az online fizetés lehetőségét. Az orvosoknak meg kell valósítani egy felületet amivel vissza tudják nézni a páciens kezeléstörténetét. Az adatokat pedig FHIR szabvány szerint kell tárolni, hogy bármilyen más egészségügyi rendszerrel kompatibilis legyen.

\noindent\textbf{A megoldási mód:}

\noindent Megismerkedtem a Django keretrendszerrel, annak felépítésével. Megterveztem a rendelő webalkalmazásának adatbázis-struktúráját, amely tartalmazza a páciensek, orvosok, kezelések és időpontfoglalások moduljait. Az alkalmazás reszponszív felhasználói felületét modern CSS megoldásokkal építettem ki, míg a backend részben a Django admin és REST API-k biztosítják az adatok hatékony kezelését. Emellett integráltam a PayPal fizetési rendszert az online fizetések lebonyolításához. A fejlesztés során a projektet GitHubon verzióztam, biztosítva ezzel a kód stabilitását és könnyű karbantarthatóságát.

\noindent\textbf{Alkalmazott eszközök, módszerek:}

\noindent Git, Github, Django, SQLite, HTML, JavaScript, CSS, Python

\noindent\textbf{Elért eredmények:}

\noindent Megismertem a Djangot, és az általa nyújtott lehetőségeket, a PayPal integrációt, a Python nyelvet, az FHIR szabványt, és a webfejlesztést. Sikeresen működik az általam integrált időpontfoglalási és fizetési rendszer.

\noindent\textbf{Kulcsszavak:}

\noindent Django, SQLite, HTML, JavaScript, CSS, Python, ORM
