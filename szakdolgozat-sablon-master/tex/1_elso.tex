\chapter{A Django keretrendszer}
\label{chap:intro}

A Django egy magas szintű Python webkeretrendszer, amely támogatja a gyors fejlesztést és az egyszerű, jól átgondolt megoldásokat. Tapasztalt fejlesztők által készített, így számos webfejlesztési nehézséget megold, és lehetővé teszi, hogy a fejlesztő alkalmazás írására koncentráljon, anélkül, hogy újra fel kellene találnia a kereket. További pozitívuma, hogy ingyenes, és nyílt forráskódú.\cite{Djangoproject}

\section{Az MVT programszervezési minta}

Django projekt lévén az alkalmazás az MVT (Model View Template) design pattern alapelveit kell, hogy kövesse. Ez áll a modellből, ahol az adatbázis struktúrájáját építjük fel, a view-ból, ami lényegében a projekt azon része, ahol a háttérfolyamatok futnak, és a template-ből, ami a felhasználói felületet tartalmazza. Ez a kapcsolata a felhasználónak az alkalmazással.

\section{Model}

A modellek a Django alkalmazáson belüli adatszerkezet kezelését és interakcióját irányítják, így a Django alkalmazások alapját képezik, mivel az adatok kritikus szerepet játszanak.
A Django Modellek egy erőteljes, Objektum-Relációs Leképezést (ORM: Object-Relational Mapping) megvalósító funkciót használnak, amely áthidalja a szakadékot a relációs adatbázis és a Python kód között. Ez a leképezés a Python objektumokat (osztályokat) adatbázis táblákká alakítja, az osztályok attribútumait oszlopokká, és az egyes példányokat a táblák soraivá.
Az ORM egyik nagy előnye, hogy lehetővé teszi az adatbázissal való interakciót Python objektumokon keresztül, anélkül, hogy SQL lekérdezéseket kellene írnunk.
A Django modellek összegzik az összes adatbázissal kapcsolatos logikát és meghatározzák az adatbázis szerkezetét, mint egy tervrajzot annak, hogy milyen adatokat szeretnénk tárolni.\cite{MVT_Design}

\section{View}

Ha az MVC modellhez szeretnénk hasonlítani, akkor az MVT modellben a View hasonló, mint az MVC-ben a Controller.
A Django view-k felelősek a felhasználói kérések feldolgozásáért és a válaszok visszaküldéséért. Híd szerepét töltik be a Model és a Template között: Adatokat gyűjtenek a modellből, logikai műveleteket (például bizonyos kritériumok alapján végzett lekérdezéseket) hajtanak végre rajtuk, majd az eredményeket átadják a Template-nek a megjelenítéshez.
A View-kat függvényekként vagy osztály alapú View-ként is megírhatjuk, attól függően, hogy az alkalmazásunk komplexitása és követelményei melyik megközelítést igénylik.\cite{MVT_Design}

\section{Template}

A Django Template-ek feladata, hogy a böngészőben megjelenítendő végső HTML kimenetet rendereljék. Meghatározzák, miként kell az adatokat bemutatni, HTML és a Django sablonnyelvének kombinációjával.
A Django sablonnyelv template tageket \verb|()| és template változókat \verb|({{ }})| alkalmaz, amelyek lehetővé teszik, hogy a sablon HTML kódjában Django módba lépjen, és így hozzáférjen a View-kban definiált változókhoz, illetve vezérlési struktúrákat használjon a megjelenítés szabályozására.
A sablonok továbbá formázhatók CSS-sel, illetve bármely kedvelt CSS keretrendszerrel, hogy a felhasználói felület mégszebb legyen. Emellett animálhatók is JS segítségével.\cite{MVT_Design}




