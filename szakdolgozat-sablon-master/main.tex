\documentclass[12pt,a4paper]{report}
\usepackage[margin=2.5cm]{geometry}
\usepackage[magyar]{babel}

% magyar nyelv tamogatas
\usepackage{t1enc}
\usepackage[T1]{fontenc}
\usepackage[utf8]{inputenc}

% A formai kovetelmenyekben megkövetelt Times betűtípus hasznalata:
\usepackage{times}

\usepackage{setspace}
\usepackage{listings,multicol}
\usepackage{xcolor}
\usepackage{textcomp}
\usepackage{enumitem}
\usepackage{booktabs}
\usepackage[unicode,hidelinks]{hyperref}
\usepackage{footnote}
\usepackage{ifthen}

% Törölhető package
\usepackage{lipsum}

% TODO csomag, amivel jól észrevehető todokat hagyhatunk a dolgozatban
\usepackage{todonotes}
% Inline TODO
\newcommand{\todoi}[1]{\todo[inline]{\textbf{TODO:} #1}}

% egyedi lablec
\usepackage{fancyhdr}
\renewcommand{\headrulewidth}{0.0pt}% Default \headrulewidth is 0.4pt

\usepackage{graphicx}
\graphicspath{{fig/}}

% Kódrészletes színezése
\definecolor{listinggray}{gray}{0.9}
\definecolor{lbcolor}{rgb}{0.9,0.9,0.9}
\lstset{
 language=Bash,
 keywordstyle=\bfseries\ttfamily\color[rgb]{0,0,1},
 identifierstyle=\ttfamily,
 commentstyle=\color[rgb]{0.133,0.545,0.133},
 stringstyle=\ttfamily\color[rgb]{0.627,0.126,0.941},
 showstringspaces=false,
 basicstyle=\scriptsize,
 numberstyle=\tiny,
 numbers=none,
 stepnumber=1,
 numbersep=10pt,
 tabsize=2,
 breaklines=true,
 prebreak = \raisebox{0ex}[0ex][0ex]{\ensuremath{\hookleftarrow}},
 breakatwhitespace=false,
 aboveskip={0.5\baselineskip},
 columns=fixed,
 upquote=true,
 extendedchars=true,
 frame=single,
 backgroundcolor=\color{lbcolor},
 literate={`}{\textasciigrave}{1}
}


\definecolor{sajatzold}{RGB}{0,105,0}

\lstdefinelanguage{diff}{
  morecomment=[f][\color{blue}]{@@},     % group identifier
  morecomment=[f][\color{red}]-,         % deleted lines
  morecomment=[f][\color{sajatzold}]+,       % added lines
  morecomment=[f][\color{magenta}]{---}, % Diff header lines (must appear after +,-)
  morecomment=[f][\color{magenta}]{+++},
}

\renewcommand{\lstlistingname}{Kódrészlet}

% Margók beállítása
\hoffset -1in
\voffset -1in
\oddsidemargin 35mm
\textwidth 150mm
\topmargin 15mm
\headheight 10mm
\headsep 5mm
\textheight 237mm

% Szerző és dolgozat adatai
%Szerző adatai
\newcommand{\nev}{Sándor Márton}
\newcommand{\szak}{programtervező informatikus BSc}
\newcommand{\tanszek}{Szoftverfejlesztés}
\newcommand{\ev}{2025}
\newcommand{\dolgozatTipusa}{Szakdolgozat}
%\newcommand{\dolgozatTipusa}{Diplomamunka}
\newcommand{\vegsoDatum}{\today}

\newcommand{\cim}{Fogászati rendelő webalkalmazás implementálása Django keretrendszerben}

%Témavezető adatai
\newcommand{\temavezetoNev}{Antal Gábor}
\newcommand{\temavezetoBeosztas}{egyetemi docens}


\begin{document}

% Másfeles sorköz
\setstretch{1.5}
\sloppy

\thispagestyle{fancy}
% A címoldalra se fej- se lábléc nem kell:
% \thispagestyle{empty}
\pagenumbering{gobble}
\begin{center}
  \vspace*{0.5cm}
  {
    \Large\bf Szegedi Tudományegyetem}

    \vspace{0.1cm}

    {\Large\bf Informatikai Intézet}

    \vspace*{4.2cm}


    {\LARGE\bf \cim}


    \vspace*{3.4cm}

    {\Large \dolgozatTipusa}

    \vspace*{3.5cm}

    %Értelemszerűen megváltoztatandó:
    {\large
    \noindent
    \begin{tabular}{@{}c@{\hspace{1cm}}c}
    \emph{Készítette:}     & \emph{Témavezető:}\\
    \bf{\nev}              & \bf{\temavezetoNev}\\
    \szak                  & \temavezetoBeosztas\\
    szakos hallgató        &
    \end{tabular}
    }

    \vspace*{2.3cm}

    {\Large
    Szeged
    \\
    \vspace{2mm}
    \ev
  }
\end{center}

\pagenumbering{arabic}
\chapter*{Feladatkiírás}
\addcontentsline{toc}{section}{Feladatkiírás}
A szakdolgozat célja, hogy a hallgató megismerkedjen az egyik legismertebb Python alapú webes keretrendszerrel, a Django-val. A megszerzett ismeretei segítségével a hallgató egy fogászati szakrendelőnek tervezett időpontfoglalási rendszert épít, amelyben elérhető a PayPal fiókos, és a bankkártyás fizetés, és az időpontfoglalás. A rendelő orvosainak lehetőségük van a saját munkaidejük bevitelére, a páciensek kezeléstörténetének megtekintésére, és a magukhoz foglalt időpontok megtekintésére. Továbbá bármelyik páciens adatait és elérhetőségeit megtekinthetik. A rendszer emellett küld emailes értesítéseket is, hogy a felhasználók és az orvosok is kapjanak információkat a foglalásokról. Az alkalmazás tartalmaz admin felületet is, amivel a felhasználónak joga van új orvosokat hozzáadni a rendszerhez, a felhasználóknak új jelszót adni, és az adatbázisban tárolt kezeléseket létrehozni, szerkeszteni, vagy törölni. A felhasznált adatbázis kezelő rendszer a Django beépített rendszere, ami sqlite3 adatbázist használ.
\lipsum[3-5]
\chapter*{Tartalmi összefoglaló}
\addcontentsline{toc}{section}{Tartalmi összefoglaló}

\noindent\textbf{A téma megnevezése:}

\noindent Fogászati rendelő webalkalmazás implementálása Django keretrendszerben

\noindent\textbf{A megadott feladat megfogalmazása:}

\noindent A Django keretrendszer által nyújtott előnyök megismerése, felhasználása a fogászati rendelő időpontfoglalási rendszerének fejlesztéséhez. Az alkalmazásnak rendelkeznie kell felhasználói, orvosi, és adminisztrátori felülettel. Meg kell valósítani az online időpontfoglalás, és az online fizetés lehetőségét. Az orvosoknak meg kell valósítani egy felületet amivel vissza tudják nézni a páciens kezeléstörténetét. Az adatokat pedig FHIR szabvány szerint kell tárolni, hogy bármilyen más egészségügyi rendszerrel kompatibilis legyen.

\noindent\textbf{A megoldási mód:}

\noindent Megismerkedtem a Django keretrendszerrel, annak felépítésével. Megterveztem a rendelő webalkalmazásának adatbázis-struktúráját, amely tartalmazza a páciensek, orvosok, kezelések és időpontfoglalások moduljait. Az alkalmazás reszponszív felhasználói felületét modern CSS megoldásokkal építettem ki, míg a backend részben a Django admin és REST API-k biztosítják az adatok hatékony kezelését. Emellett integráltam a PayPal fizetési rendszert az online fizetések lebonyolításához. A fejlesztés során a projektet GitHubon verzióztam, biztosítva ezzel a kód stabilitását és könnyű karbantarthatóságát.

\noindent\textbf{Alkalmazott eszközök, módszerek:}

\noindent Git, Github, Django, SQLite, HTML, JavaScript, CSS, Python

\noindent\textbf{Elért eredmények:}

\noindent Megismertem a Djangot, és az általa nyújtott lehetőségeket, a PayPal integrációt, a Python nyelvet, az FHIR szabványt, és a webfejlesztést. Sikeresen működik az általam integrált időpontfoglalási és fizetési rendszer.

\noindent\textbf{Kulcsszavak:}

\noindent Django, SQLite, HTML, JavaScript, CSS, Python, ORM

\addcontentsline{toc}{section}{Tartalomjegyzék}

\thispagestyle{plain}
\tableofcontents

\pagestyle{fancy}
\fancyhf{}
\fancyhead[L]{\textit{\cim}}
\fancyfoot[R]{\thepage}
\renewcommand{\headrulewidth}{0.4pt}% Default \headrulewidth is 0.4pt

\chapter{A Django keretrendszer}
\label{chap:intro}

A Django egy magas szintű Python webkeretrendszer, amely támogatja a gyors fejlesztést és az egyszerű, jól átgondolt megoldásokat. Tapasztalt fejlesztők által készített, így számos webfejlesztési nehézséget megold, és lehetővé teszi, hogy a fejlesztő alkalmazás írására koncentráljon, anélkül, hogy újra fel kellene találnia a kereket. További pozitívuma, hogy ingyenes, és nyílt forráskódú.\cite{Djangoproject}

\section{Az MVT programszervezési minta}

Django projekt lévén az alkalmazás az MVT (Model View Template) design pattern alapelveit kell, hogy kövesse. Ez áll a model-ből, ahol az adatbázis struktúrájáját építjük fel, a view-ból, ami lényegében a projekt azon része, ahol a háttérfolyamatok futnak, és a template-ből, ami a felhasználói felületet tartalmazza. Ez a kapcsolata a felhasználónak az alkalmazással.

\section{Model}

A Model-ek a Django alkalmazáson belüli adatszerkezet kezelését és interakcióját irányítják, így a Django alkalmazások alapját képezik, mivel az adatok kritikus szerepet játszanak.
A Django Model-ek egy erőteljes, Objektum-Relációs Leképezést (ORM: Object-Relational Mapping) megvalósító funkciót használnak, amely áthidalja a szakadékot a relációs adatbázis és a Python kód között. Ez a leképezés a Python objektumokat (osztályokat) adatbázis táblákká alakítja, az osztályok attribútumait oszlopokká, és az egyes példányokat a táblák soraivá.
Az ORM egyik nagy előnye, hogy lehetővé teszi az adatbázissal való interakciót Python objektumokon keresztül, anélkül, hogy SQL lekérdezéseket kellene írnunk.
A Django Model-ek összegzik az összes adatbázissal kapcsolatos logikát és meghatározzák az adatbázis szerkezetét, mint egy tervrajzot annak, hogy milyen adatokat szeretnénk tárolni.\cite{MVT_Design}

\section{View}

Ha az MVC modellhez szeretnénk hasonlítani, akkor az MVT modellben a View hasonló, mint az MVC-ben a Controller.
A Django view-k felelősek a felhasználói kérések feldolgozásáért és a válaszok visszaküldéséért. Híd szerepét töltik be a Model és a Template között: Adatokat gyűjtenek a Model-ből, logikai műveleteket (például bizonyos kritériumok alapján végzett lekérdezéseket) hajtanak végre rajtuk, majd az eredményeket átadják a Template-nek a megjelenítéshez.
A View-kat függvényekként vagy osztály alapú View-ként is megírhatjuk, attól függően, hogy az alkalmazásunk komplexitása és követelményei melyik megközelítést igénylik.\cite{MVT_Design}

\section{Template}

A Django Template-ek feladata, hogy a böngészőben megjelenítendő végső HTML kimenetet rendereljék. Meghatározzák, miként kell az adatokat bemutatni, HTML és a Django sablonnyelvének kombinációjával.
A Django sablonnyelv template tageket \verb|()| és template változókat \verb|({{ }})| alkalmaz, amelyek lehetővé teszik, hogy a sablon HTML kódjában Django módba lépjen, és így hozzáférjen a View-kban definiált változókhoz, illetve vezérlési struktúrákat használjon a megjelenítés szabályozására.
A sablonok továbbá formázhatók CSS-sel, illetve bármely kedvelt CSS keretrendszerrel, hogy a felhasználói felület mégszebb legyen. Emellett animálhatók is JS segítségével.\cite{MVT_Design}





\chapter{Alkalmazás struktúrája}
\label{chap:fejezet2}

Egy Django projekt esetében a projekt felépítése modulárisan, egy vagy több alkalmazásból (app) áll, melyek mindegyike egy adott funkcionális területért felel. A szakdolgozatom esetében a "rendelo" mappa tartalmazza a teljes webalkalmazás forráskódját.\\
A "rendelo" mappa a következő részekből áll:

\begin{itemize}
	\item Gyökérszint:
	\begin{itemize}
		\item manage.py: A Django projekt parancssori kezelője, amely a fejlesztési feladatok (például migrációk futtatása, szerver indítása) végrehajtását segíti.
		\item db.sqlite3: Az alapértelmezett, fejlesztési környezetben használt SQLite adatbázis fájlja.
	\end{itemize}
	\item Projekt főkönyvtára ("rendelo"): Itt találhatók a projekt globális beállításait és konfigurációs fájljait, mint például a settings.py, urls.py, wsgi.py és asgi.py. Ezek a fájlok felelősek az alkalmazás működésének alapvető paramétereinek meghatározásáért, az útvonalak kezeléséért és a szerverrel való kommunikációért.
	\item Alkalmazás könyvtára ("rendeloweboldal"): Ez a rész tartalmazza a rendszer egyes moduljait, amelyek a következő fő komponensekből állnak:
	\begin{itemize}
		\item models.py: Az adatbázis szerkezetét definiáló modellek, melyek meghatározzák a páciensek, orvosok, kezelések, időpontfoglalások és az időpont foglalások fizetési státuszának struktúráját.
		\item views.py: A felhasználói kérések feldolgozásáért és az üzleti logika megvalósításáért felelős fájl, amely összeköti a modelleket a sablonokkal.
		\item forms.py: Az űrlapok és azok validációs szabályainak definíciója, melyek révén az adatbevitel és ellenőrzés történik.
		\item urls.py: Az alkalmazás specifikus URL-konfigurációja, amely a különböző view-k elérését biztosítja.
		\item admin.py: A Django beépített adminisztrátori felület konfigurációját tartalmazza. Hozzá kell adni az összes modellt az adatbázisból, amit elérhetővé szeretnénk tenni rajta.
		\item migrations/: Az adatbázis változásait követő migrációs fájlokat tartalmazza, dokumentálva a modellek módosításait.
		\item static/ és templates/: A statikus fájlokat (CSS, JavaScript, és az alkalmazás designjához tartozó képek) illetve a HTML template-eket rendszerezi, amelyek a felhasználói felület megjelenítéséért felelősek.
	\end{itemize}
\end{itemize}

A projekt kialakítása moduláris és átlátható, mely lehetővé teszi a fejlesztés, karbantartás és bővítés egyszerű kezelését.
Emellett a projekt verziókezelése a GitHubon történik, így könnyen nyomon követhető az egész alkalmazás fejlesztése.
\chapter{Összefoglaló}
\label{chap:conclusion}

\lipsum[50-53]
\chapter*{Nyilatkozat}
\addcontentsline{toc}{section}{Nyilatkozat}

% A szövegben a dolgozat típusa alapján "diplomamunkámat" vagy "szakdolgozatomat"
% szövegrészt készíti el. 
\ifthenelse{\equal{\dolgozatTipusa}{Diplomamunka}}
  {% True case
   \newcommand{\dolgozatomat}{diplomamunkámat}%
  }
  {% false case
   \newcommand{\dolgozatomat}{szakdolgozatomat}%
  }
%

\noindent Alulírott \nev{} \szak{} szakos hallgató, kijelentem, hogy a dolgozatomat a Szegedi Tudományegyetem, Informatikai Intézet \tanszek{} Tanszékén készítettem, \szak{} diploma megszerzése érdekében. 

Kijelentem, hogy a dolgozatot más szakon korábban nem védtem meg, saját munkám eredménye, és csak a hivatkozott forrásokat (szakirodalom, eszközök, stb.) használtam fel. 

Tudomásul veszem, hogy \dolgozatomat{} a Szegedi Tudományegyetem Informatikai Intézet könyvtárában, a helyben olvasható könyvek között helyezik el.

\vspace*{2cm}

\begin{table}[!h]
  \begin{tabular}{lc}
    Szeged, \vegsoDatum{} \hspace{2cm}  & \makebox[7cm]{\dotfill} \\
                                        & aláírás \\
  \end{tabular}
\end{table}

\thispagestyle{plain}
\addcontentsline{toc}{chapter}{Köszönetnyílvánítás}
\chapter*{Köszönetnyílvánítás}

\lipsum[61]

\addcontentsline{toc}{chapter}{Irodalomjegyzék}
\bibliographystyle{plain}
\bibliography{references}

\end{document}
